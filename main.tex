\documentclass[a4paper,12pt]{report}
\usepackage{graphicx}
\usepackage{hyperref}
\usepackage[T1]{fontenc}
\usepackage[utf8]{inputenc}
\usepackage[english,italian]{babel}
\usepackage{setspace}
\usepackage[paper=a4paper,margin=1in]{geometry}
\usepackage{lipsum}
\usepackage{microtype}
\usepackage{times}
\usepackage{xcolor}
\usepackage{hyperref}
\usepackage{float}
\usepackage{subfiles}
\usepackage[backend=biber, sorting=none]{biblatex}
\usepackage{csquotes}
\usepackage{tikz}
\usepackage{listings}
\usepackage{listingsutf8}
\usepackage{forest}
\usepackage{amssymb}
\usepackage{pifont}
\usepackage{array}
\graphicspath{ {images/}, {diagrams/}, {screens/} }

\addbibresource{extra/bibliografy.bib}

\title{DevOps: studio e implementazione di una pipeline di CI e CD nel progetto Sphere}
\author{Simone Renzo}
\date{Febbraio 2021}

%% Folder Tree
\definecolor{folderbg}{RGB}{124,166,198}
\definecolor{folderborder}{RGB}{110,144,169}

\def\Size{4pt}
\tikzset{
  folder/.pic={
    \filldraw[draw=folderborder,top color=folderbg!50,bottom color=folderbg]
      (-1.05*\Size,0.2\Size+5pt) rectangle ++(.75*\Size,-0.2\Size-5pt);  
    \filldraw[draw=folderborder,top color=folderbg!50,bottom color=folderbg]
      (-1.15*\Size,-\Size) rectangle (1.15*\Size,\Size);
  }
}

%% Code Snippets
\definecolor{codegreen}{rgb}{0,0.6,0}
\definecolor{codegray}{rgb}{0.5,0.5,0.5}
\definecolor{codepurple}{rgb}{0.58,0,0.82}
\definecolor{backcolour}{rgb}{0.95,0.95,0.92}

\lstdefinestyle{custom_style} {
    backgroundcolor=\color{backcolour},   
    commentstyle=\color{codegreen},
    keywordstyle=\color{magenta},
    numberstyle=\tiny\color{codegray},
    stringstyle=\color{codepurple},
    %basicstyle=\ttfamily\footnotesize,
    %breakatwhitespace=false,         
    breaklines=true,                 
    %captionpos=b,                    
    %keepspaces=true,                 
    numbers=left,                    
    numbersep=5pt,                  
    %showspaces=false,                
    %showstringspaces=false,
    %showtabs=false,                  
    tabsize=2
}
\lstset{style=custom_style}

%% Ini files
\lstdefinelanguage{Ini} {
    basicstyle=\ttfamily\small,
    columns=fullflexible,
    morecomment=[s][\color{blue}\bfseries]{[}{]},
    morecomment=[l]{\#},
    morecomment=[l]{;},
    commentstyle=\color{gray}\ttfamily,
    morekeywords={},
    otherkeywords={=,:},
    keywordstyle={\color{green}\bfseries}
}

%% Dockerfiles
\lstdefinelanguage{docker} {
  keywords={FROM, RUN, COPY, ADD, ENTRYPOINT, CMD,  ENV, ARG, WORKDIR, EXPOSE, LABEL, USER, VOLUME, STOPSIGNAL, ONBUILD, MAINTAINER},
  keywordstyle=\color{blue}\bfseries,
  identifierstyle=\color{black},
  sensitive=false,
  comment=[l]{\#},
  commentstyle=\color{purple}\ttfamily,
  stringstyle=\color{red}\ttfamily,
  morestring=[b]',
  morestring=[b]"
}

\lstset {
  basicstyle=\ttfamily,
  showstringspaces=false,
  commentstyle=\color{red},
  keywordstyle=\color{blue},
  inputencoding=utf8,
  extendedchars=true
}

%% Yaml
\makeatletter
\newcommand\YAMLcolonstyle{\color{red}\mdseries}
\newcommand\YAMLkeystyle{\color{black}\bfseries}
\newcommand\YAMLvaluestyle{\color{blue}\mdseries}

\newcommand\language@yaml{yaml}

\expandafter\expandafter\expandafter\lstdefinelanguage
\expandafter{\language@yaml}
{
  keywords={true,false,null,y,n},
  keywordstyle=\color{darkgray}\bfseries,
  basicstyle=\YAMLkeystyle,
  sensitive=false,
  comment=[l]{\#},
  morecomment=[s]{/*}{*/},
  commentstyle=\color{purple}\ttfamily,
  stringstyle=\YAMLvaluestyle\ttfamily,
  moredelim=[l][\color{orange}]{\&},
  moredelim=[l][\color{magenta}]{*},
  moredelim=**[il][\YAMLcolonstyle{:}\YAMLvaluestyle]{:},
  morestring=[b]',
  morestring=[b]",
  literate = {---}{{\ProcessThreeDashes}}3
             {>}{{\textcolor{red}\textgreater}}1     
             {|}{{\textcolor{red}\textbar}}1 
             {\ -\ }{{\mdseries\ -\ }}3,
}

\lst@AddToHook{EveryLine}{\ifx\lst@language\language@yaml\YAMLkeystyle\fi}
\makeatother
\newcommand\ProcessThreeDashes{\llap{\color{cyan}\mdseries-{-}-}}

%% Groovy
\definecolor{groovyblue}{HTML}{0000A0}
\definecolor{groovygreen}{HTML}{008000}
\definecolor{darkgray}{rgb}{.4,.4,.4}
 
\lstdefinelanguage{Groovy}[]{Java}{
  keywordstyle=\color{groovyblue}\bfseries,
  stringstyle=\color{groovygreen}\ttfamily,
  keywords=[3]{each, findAll, groupBy, collect, inject, eachWithIndex},
  morekeywords={def, as, in, use},
  moredelim=[is][\textcolor{darkgray}]{\%\%}{\%\%},
  moredelim=[il][\textcolor{darkgray}]{§§}
}

%% Checkmark and Cross
\newcommand{\cmark}{\ding{51}}%
\newcommand{\xmark}{\ding{55}}%

%%%%********************************************************************
% fancy quotes
\definecolor{quotemark}{gray}{0.7}
\makeatletter
\def\fquote{%
	\@ifnextchar[{\fquote@i}{\fquote@i[]}%]
}%
\def\fquote@i[#1]{%
	\def\tempa{#1}%
	\@ifnextchar[{\fquote@ii}{\fquote@ii[]}%]
}%
\def\fquote@ii[#1]{%
	\def\tempb{#1}%
	\@ifnextchar[{\fquote@iii}{\fquote@iii[]}%]
}%
\def\fquote@iii[#1]{%
	\def\tempc{#1}%
	\vspace{1em}%
	\noindent%
	\begin{list}{}{%
			\setlength{\leftmargin}{0.1\textwidth}%
			\setlength{\rightmargin}{0.1\textwidth}%
		}%
		\item[]%
		\begin{picture}(0,0)%
			\put(-15,-5){\makebox(0,0){\scalebox{3}{\textcolor{quotemark}{``}}}}%
		\end{picture}%
		\begingroup\itshape}%
	%%%%********************************************************************
	\def\endfquote{%
		\endgroup\par%
		\makebox[0pt][l]{%
			\hspace{0.8\textwidth}%
			\begin{picture}(0,0)(0,0)%
				\put(15,15){\makebox(0,0){%
						\scalebox{3}{\color{quotemark}''}}}%
		\end{picture}}%
		\ifx\tempa\empty%
		\else%
		\ifx\tempc\empty%
		\hfill\rule{100pt}{0.5pt}\\\mbox{}\hfill\tempa,\ \emph{\tempb}%
		\else%
		\hfill\rule{100pt}{0.5pt}\\\mbox{}\hfill\tempa,\ \emph{\tempb},\ \tempc%
		\fi\fi\par%
		\vspace{0.5em}%
	\end{list}%
}%
\makeatother
%%%%********************************************************************

%%% Dedication
\newenvironment{dedication}
  {\clearpage 
   \thispagestyle{empty}
   \vspace*{\stretch{1}}
   \itshape
   \raggedleft
  }
  {\par
   \vspace{\stretch{3}}
   \clearpage
  }
%%%

\begin{document}
	
	\begin{titlepage}
		\noindent
		\begin{minipage}[t]{0.19\textwidth}
			\vspace{-4mm}{\includegraphics[scale=1.15]{logo_unimib.pdf}}
		\end{minipage}
		\begin{minipage}[t]{0.81\textwidth}
			{
				\setstretch{1.42}
				{\textsc{Università degli Studi di Milano - Bicocca}} \\
				\textbf{Scuola di Scienze} \\
				\textbf{Dipartimento di Informatica, Sistemistica e Comunicazione} \\
				\textbf{Corso di laurea in Informatica} \\
				\par
			}
		\end{minipage}
		
		\vspace{40mm}
		
		\begin{center}
			{\LARGE{
					\setstretch{1.2}
					\textbf{DevOps: studio e implementazione \\ di una pipeline di CI e CD \\ nel progetto Sphere}
					\par
			}}
		\end{center}
		
		\vspace{40mm}
		
		\noindent
		{\large \textbf{Relatore:} Prof. Mariani Leonardo} \\
		
		\noindent
		{\large \textbf{Co-relatore:} Dott. Mesiano Cristian}
		
		\vspace{15mm}
		
		\begin{flushright}
			{\large \textbf{Relazione della prova finale di:}} \\
			\large{Renzo Simone} \\
			\large{Matricola 781616} 
		\end{flushright}
		
		\vspace{40mm}
		\begin{center}
			{\large{\bf Anno Accademico 2019-2020}}
		\end{center}
		
		\restoregeometry
		
	\end{titlepage}
	
	\clearpage\null\newpage
	
	\begin{dedication}
	    Ai miei genitori, scusandomi per l'enorme ritardo.\\
	    A Donata, per essermi stata sempre a fianco in un anno veramente difficile.\\
	    Ai miei amici, per aver gufato sempre e comunque.\\
	    Ai miei colleghi, grazie al quale non sarebbe stata possibile questa opportunità.\\
	    A me stesso, per non aver smesso di crederci.
	   
	    \vspace{4mm}
	    Un ringraziamento particolare al Dott. Cristian Mesiano per aver creduto\\ed investito in me fin dall'inizio.
	\end{dedication}
	
	\selectlanguage{english}
	\begin{abstract}
		In un mondo in continuo sviluppo, la necessità di adattarsi velocemente al cambiamento è spesso ciò che permette di contraddistinguere le realtà di successo dalle fallimentari. Il software è probabilmente uno dei prodotti che più segue questa filosofia di cambiamento repentino, con il rilascio continuo di nuove tecnologie, nuove metodiche e la conseguente necessità di cambiare spesso rotta e requisiti in base alle necessità o al mercato di riferimento.\\*
		
		La nascita di metodi \emph{Agile} e di nuove filosofie improntate all'unire ciò che prima era separato, in un unico processo, hanno permesso di adattarsi con successo ai cambiamenti, rendendo l'industria del software quella più all'avanguardia e resiliente nel tempo, continuando tutt'oggi a migliorarsi sempre più.
		
 		\begin{fquote}[Leon C. Megginson][1963][citando C. Darwin]
		 	It is not the strongest of the species that survives, nor the most intelligent that survives.
		 	It is the one that is the most adaptable to change.
	 	\end{fquote}
	\end{abstract}

	\selectlanguage{italian}
	
	\tableofcontents
	
	\listoffigures
    
    \listoftables
	
	\subfile{chapters/1_intro}
	
	\subfile{chapters/2_dev_models}
	
	\subfile{chapters/3_cloud_auto}
	
	\subfile{chapters/4_process_analysis}
	
	\subfile{chapters/5_tech_back}
	
	\subfile{chapters/6_cloud_arch}
	
	\subfile{chapters/7_ci_pipeline}
	
	\subfile{chapters/8_cd_pipeline}
	
	\subfile{chapters/9_final}
	
	\clearpage\null\newpage
	
	\printbibliography[heading=bibintoc]
	
\end{document}