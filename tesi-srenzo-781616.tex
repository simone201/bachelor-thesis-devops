\documentclass[a4paper,12pt]{report}
\usepackage{graphicx}
\usepackage{hyperref}
\usepackage[T1]{fontenc}
\usepackage[utf8]{inputenc}
\usepackage[english,italian]{babel}
\usepackage{setspace}
\usepackage[paper=a4paper,margin=1in]{geometry}
\usepackage{lipsum}
\usepackage{microtype}
\usepackage{times}
\usepackage{xcolor}
\usepackage{hyperref}
\graphicspath{ {images/} }

\title{Test Document}
\author{Simone Renzo}
\date{Febbraio 2021}

%%%%********************************************************************
% fancy quotes
\definecolor{quotemark}{gray}{0.7}
\makeatletter
\def\fquote{%
	\@ifnextchar[{\fquote@i}{\fquote@i[]}%]
}%
\def\fquote@i[#1]{%
	\def\tempa{#1}%
	\@ifnextchar[{\fquote@ii}{\fquote@ii[]}%]
}%
\def\fquote@ii[#1]{%
	\def\tempb{#1}%
	\@ifnextchar[{\fquote@iii}{\fquote@iii[]}%]
}%
\def\fquote@iii[#1]{%
	\def\tempc{#1}%
	\vspace{1em}%
	\noindent%
	\begin{list}{}{%
			\setlength{\leftmargin}{0.1\textwidth}%
			\setlength{\rightmargin}{0.1\textwidth}%
		}%
		\item[]%
		\begin{picture}(0,0)%
			\put(-15,-5){\makebox(0,0){\scalebox{3}{\textcolor{quotemark}{``}}}}%
		\end{picture}%
		\begingroup\itshape}%
	%%%%********************************************************************
	\def\endfquote{%
		\endgroup\par%
		\makebox[0pt][l]{%
			\hspace{0.8\textwidth}%
			\begin{picture}(0,0)(0,0)%
				\put(15,15){\makebox(0,0){%
						\scalebox{3}{\color{quotemark}''}}}%
		\end{picture}}%
		\ifx\tempa\empty%
		\else%
		\ifx\tempc\empty%
		\hfill\rule{100pt}{0.5pt}\\\mbox{}\hfill\tempa,\ \emph{\tempb}%
		\else%
		\hfill\rule{100pt}{0.5pt}\\\mbox{}\hfill\tempa,\ \emph{\tempb},\ \tempc%
		\fi\fi\par%
		\vspace{0.5em}%
	\end{list}%
}%
\makeatother
%%%%********************************************************************

\begin{document}
	
	\begin{titlepage}
		\noindent
		\begin{minipage}[t]{0.19\textwidth}
			\vspace{-4mm}{\includegraphics[scale=1.15]{logo_unimib.pdf}}
		\end{minipage}
		\begin{minipage}[t]{0.81\textwidth}
			{
				\setstretch{1.42}
				{\textsc{Università degli Studi di Milano - Bicocca}} \\
				\textbf{Scuola di Scienze} \\
				\textbf{Dipartimento di Informatica, Sistemistica e Comunicazione} \\
				\textbf{Corso di laurea in Informatica} \\
				\par
			}
		\end{minipage}
		
		\vspace{40mm}
		
		\begin{center}
			{\LARGE{
					\setstretch{1.2}
					\textbf{DevOps: studio e implementazione \\ di una pipeline di CI e CD \\ nel progetto Sphere}
					\par
			}}
		\end{center}
		
		\vspace{40mm}
		
		\noindent
		{\large \textbf{Relatore:} Prof. Mariani Leonardo} \\
		
		\noindent
		{\large \textbf{Tutor Aziendale:} Dott. Mesiano Cristian}
		
		\vspace{15mm}
		
		\begin{flushright}
			{\large \textbf{Relazione della prova finale di:}} \\
			\large{Renzo Simone} \\
			\large{Matricola 781616} 
		\end{flushright}
		
		\vspace{40mm}
		\begin{center}
			{\large{\bf Anno Accademico 2020-2021}}
		\end{center}
		
		\restoregeometry
		
	\end{titlepage}
	
	\selectlanguage{english}
	\begin{abstract}
		In un mondo in continuo sviluppo, la necessità di adattarsi velocemente al cambiamento è spesso
		ciò che permette di contraddistinguere realtà di successo dalle fallimentari.
		Il software è probabilmente uno dei prodotti che più segue questa filosofia di cambiamento repentino,
		con la continua uscita di nuove tecnologie, nuove metodiche e la conseguente necessità di cambiare spesso
		rotta e requisiti in base alle necessità o al mercato di riferimento.
		
		La nascita di metodi \emph{Agile} e di nuove filosofie improntate all'unire ciò che prima era separato,
		in un unico processo, hanno permesso di adattarsi con successo ai cambiamenti, rendendo l'industria del
		software quella più all'avanguardia e resiliente nel tempo, continuando tutt'oggi a migliorarsi sempre più.
		
 		\begin{fquote}[Charles Darwin]
		 	It is not the strongest of the species that survives, nor the most intelligent that survives.
		 	It is the one that is the most adaptable to change.
	 	\end{fquote}
	\end{abstract}

	\selectlanguage{italian}
	\tableofcontents
	
	\chapter{Introduzione}
	
	\section{Il Contesto Aziendale}
	
		\subsection{Storia}
		
			\lipsum[1]
		
		\subsection{Il progetto Sphere}
		
			\lipsum[2]
	
		\subsection{Profilo personale in azienda}
		
			In PerceptoLab Srl, il candidato è stato assunto in Giugno 2020, con ruolo di Infrastructure \& DevOps Engineer,
			integrato nel team di Backend Development.
	
	\section{Scopo del Project Work}
	
		\subsection{Obiettivi}
	
			Il Project Work ha come scopo il design e l'implementazione di un processo che includa una Pipeline di Continuous
			Integration e di Continuous Delivery, nell'ambito del progetto Sphere.
			
			In particolare si prefigge questi obiettivi:
			\begin{enumerate}
				\item Creazione di un processo di \emph{Continuous Integration} per il repository progettuale, mediante l'uso di tool per il testing automatico e per la gestione delle Pull Request, con integrazione per build in ambiente macOS;
				\item Creazione di un processo di \emph{Continuous Delivery} per la creazione di immagini Docker mediante utilizzo di tag specifici su repository e delivery degli artefatti su registry remoto;
				\item Integrazione nella pipeline DevOps di \emph{analisi statica e dinamica} del codice mediante tools dedicati e definizione di quality gates in base alle necessità progettuali;
				\item Creazione e gestione dell'infrastruttura (basata su \emph{Amazon Web Services} con tecnologia \emph{Kubernetes}) necessaria al deployment dei servizi sviluppati nel progetto aziendale.
			\end{enumerate}
	
		\subsection{Pianificazione del Lavoro}
	
			Il Project Work si è svolto durante il periodo di 3 mesi tra l'1 Ottobre 2020 ed il 31 Dicembre 2020, in modalità di remote
			working con l'utilizzo di tools di collaboration integrati in Google GSuite (Google Chat, Meets) e nella suite Atlassian
			(Bitbucket, Jira, Confluence).
			
			INSERIRE UN GANTT.
	
		\subsection{Prodotti Finali}
		
			I prodotti del project work saranno i seguenti:
			\begin{itemize}
				\item Analisi dei Requisiti per i processi da implementare
				\item Pipeline di Continuous Integration per i servizi di Backend e Mobile
				\item Pipeline di Continuous Delivery per i servizi di Backend (Docker Containers)
				\item Quality Assurance Gates basati sulla analisi dei test e del codice con tools dedicati
				\item Infrastruttura basata su Amazon Web Services (Risorse Cloud, VMs) e Kubernetes su AWS (EKS) per gestire i processi descritti.
			\end{itemize}
	
	\chapter{Modelli di Sviluppo del Software}
	
		\section{Il ciclo di vita del software}
	
			\begin{figure}[h]
				\centering
				\includegraphics[width=0.4\textwidth]{sdlc}
				\caption{Software Development Lifecycle}
				\label{fig:sdlc}
			\end{figure}
	
			\subsection{Le fasi del ciclo di vita}
	
				Il ciclo di vita del software si riferisce ad una metodologia che permetta di ottenere software di massima qualità
				al minimo costo di produzione e nel minor tempo possibile, dividendo la sua vita in diverse fasi consequenziali:
				\begin{enumerate}
					\item Planning e Analisi
					\item Design
					\item Implementazione
					\item Testing e Integrazione
					\item Manutenzione
				\end{enumerate}
				
				\paragraph{Planning e Analisi}
				Si analizzano i sistemi esistenti per i cambiamenti necessari ed il problema da risolvere in termini di software development.
				Questa fase crea in output una serie di \emph{Requisiti} che possono essere Funzionali, Non Funzionali, o di Dominio, ed un piano
				di lavoro per sviluppare tali requisiti in un tempo definito (ma, come vedremo, variabile in metodi \emph{Agile}).
				Le definizioni di tali requisiti sono dettate dallo standard \href{https://standards.ieee.org/standard/610_12-1990.html}{\emph{IEEE 610.12-1990}}.
				
				\paragraph{Design}
				I requisiti vengono trasformati in una \emph{specifica di Design} (architetturale ed implementativa), che verrà in seguito analizzata
				dagli \emph{stakeholders}, ottenendo così feedback e suggerimenti in base alle esigenze. In questa fase diventa cruciale implementare
				un sistema per incorporare i feedback così da migliorare il design finale ed evitare costi aggiuntivi a fine sviluppo.
				
				\paragraph{Implementazione}
				Questa fase inizia lo sviluppo del software in se, seguendo la specifica di design della fase precedente, ed utilizzando convenzioni,
				code style, pratiche e linee guida comuni per tutti i soggetti coinvolti nello sviluppo. L'utilizzo di linee guida comuni permette di
				evitare fraintendimenti all'interno del team di sviluppo, e di facilitare le fasi future di manutenzione.
				
				\paragraph{Testing e Integrazione}
				Il software sviluppato viene sottoposto a test per difetti e mancanze, risolvendo i problemi trovati lungo il percorso e migliorando
				le feature implementate fino ad arrivare ad una qualità in linea con le specifiche originali.
				In seguito, viene integrato con il resto dell'ambiente mediante deployment, così da poterlo iniziare ad utilizzare in casi reali.
				
				\paragraph{Manutenzione}
				Alla fine del processo, difficilmente si saranno raggiunti tutti i requisiti alla perfezione, motivo per cui la fase di manutenzione
				gioca un ruolo fondamentale per gestire tutto ciò che segue lo sviluppo principale del software. Questa fase permette quindi di analizzare
				i comportamenti sul campo del software sviluppato, così da agire di conseguenza nel risolvere problemi in modo più mirato.
	
			\subsection{I modelli classici: Waterfall e Iterativo}
			
				\begin{figure}[h]
					\centering
					\includegraphics[width=0.8\textwidth]{waterfall}
					\caption{Modello Waterfall}
					\label{fig:waterfall}
				\end{figure}
				
				Il processo di sviluppo storicamente più tradizionale e semplice è chiamato \textbf{Waterfall}.
				Il nome suggerisce come, rispetto alle fasi del ciclo di vita del software, queste vengano eseguite in "cascata",
				dove la fine di una fase permette di iniziare quella successiva, seguendo ciò che era stato appreso dalla produzione
				manufatturiera applicandolo in ambito dello sviluppo software.
				
				La creazione di tale processo ha permesso di superare i limiti del processo \emph{code and fix}, permettendo di pianificare
				in modo più strutturato e dividendo in modo netto le problematiche in base alla fase di appartenenza.
				Altrettante sono però state le problematiche derivanti dalla sua applicazione, tra cui:
				\begin{itemize}
					\item Le fasi di \emph{alpha/beta} testing ripercorrono per natura tutte le fasi del processo, rallentando lo sviluppo;
					\item Ogni fase viene congelata dopo la sua fine, rendendo impossibile la comunicazione tra clienti e sviluppatori dopo la fase iniziale;
					\item La pianificazione viene effettuata solo all'inizio, orientando lo sviluppo ad una data specifica di rilascio; Ogni errore porta a ritardare tale data, che non può però essere stimata di nuovo;
					\item La stima dei costi e delle risorse si rende difficile senza la prima fase di Analisi;
					\item La specifica di requisiti vincola il prodotto da sviluppare, mentre nei casi reali spesso le necessità del cliente cambiano in corso d'opera, specialmente sul lungo termine;
				\end{itemize}
	
				\begin{figure}[h]
					\centering
					\includegraphics[width=0.6\textwidth]{iterative}
					\caption{Modello Iterativo}
					\label{fig:iterative}
				\end{figure}
	
				Un altro processo.
	
		\section{Modelli \emph{Agile}}
	
			\subsection{La filosofia \emph{Agile}}
	
			\subsection{Un modello di successo: SCRUM Framework}
			
			\subsection{Un modello veloce: eXtreme Programming}
	
		\section{Metodologie di Sviluppo \emph{Agile}: DevOps}
	
			\subsection{Un metodo ed un'etica}
			
			\subsection{Obiettivi delle pratiche DevOps}
			
			\subsection{Processo di Riferimento}
	
				\subsubsection{Continuous Integration}
				
				\subsubsection{Continuous Delivery}
				
				\subsubsection{Continuous Deployment}
	
		\section{Il ROI del DevOps}
	
	\chapter{Sviluppo ed Automazione su Cloud}
	
	\section{Perchè il Cloud?}
	
	\section{\emph{Infrastructure-as-a-Code}}
	
	\section{\emph{Configuration-as-a-Code}}
	
	\section{Containers ed ambienti controllati}
	
	\chapter{Analisi del Processo di Sviluppo}
	
	\section{Struttura del Progetto}
	
	\section{Il processo di sviluppo}
	
	\section{Requisiti e KPI}
	
	\section{Il processo DevOps}
	
	\subsection{Architettura High-Level e Fasi}
	
	\subsection{Gestione del Codice}
	
	\subsection{Pull Requests e Code Review}
	
	\subsection{Continuous Integration}
	
	\subsubsection{Test Automation}
	
	\subsubsection{Code Analysis}
	
	\subsection{Continuous Delivery}
	
	\subsubsection{Build Automation}
	
	\subsubsection{Artifacts Delivery}
	
	\subsection{Deployment}
	
	\section{Tecnologie e Strumenti}
	
	\subsection{SCM: Git}
	
	\subsection{Build System: Bazel}
	
	\subsection{Cloud Provider: AWS}
	
	\subsection{PR Management: Phabricator ed Arcanist}
	
	\subsection{CI/CD: Jenkins}
	
	\subsection{Code Analysis: SonarQube}
	
	\chapter{Tecnologie di Background}
	
	\section{Cloud Provider: AWS}
	
	\section{IaaC: Terraform}
	
	\section{CaaC: Ansible}
	
	\section{Containers: Docker}
	
	\chapter{Architettura Cloud}
	
	\section{Diagramma Architetturale}
	
	\section{Configurazione di Phabricator}
	
	\section{Configurazione di Jenkins}
	
	\subsection{Agent su AWS EC2}
	
	\subsection{Agent macOS On-Premise}
	
	\section{Configurazione di SonarQube}
	
	\section{Creazione ambiente di build con Docker}
	
	\chapter{La Pipeline di CI}
	
	\section{Tecnologie e Strumenti}
	
	\subsection{Phabricator ed Arcanist: un flusso controllato}
	
	\subsection{Jenkins: il motore del processo}
	
	\subsection{Docker: ambiente di testing unificato}
	
	\subsection{SonarQube: controllo qualità}
	
	\section{Le fasi della Pipeline}
	
	\section{Analisi del Codice}
	
	\subsection{Quality Gates}
	
	\subsection{Code Coverage}
	
	\subsection{Risposta a cambiamenti nella qualità}
	
	\section{Risultati di Testing e QA}
	
	\chapter{La Pipeline di CD}
	
	\section{Tecnologie e Strumenti}
	
	\subsection{Jenkins: il motore del processo}
	
	\subsection{Docker: ambiente di build unificato}
	
	\section{Le fasi della Pipeline}
	
	\section{Analisi delle Vulnerabilità}
	
	\section{Risultati}
	
	\chapter{Conclusioni}
	
	\section{Obiettivi Raggiunti}
	
	\section{Risultati su Requisiti e KPI}
	
	\section{Evoluzioni Future}
	
	\section{Considerazioni Personali}
	
\end{document}